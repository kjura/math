\documentclass[a4paper,oneside,openright,11pt]{article}
\usepackage[utf8]{inputenc}
\usepackage[T1]{fontenc}
\usepackage{amsmath}
\usepackage{amsthm}
\usepackage{amsfonts}


%\usepackage[a4paper, total={6in, 8in}]{geometry}

\textwidth\paperwidth
\advance\textwidth -55mm
\oddsidemargin-1in
\advance\oddsidemargin 30mm
\evensidemargin-1in
\advance\evensidemargin 25mm
\topmargin -1in
\advance\topmargin 2cm
\setlength\textheight{48\baselineskip}
\addtolength\textheight{\topskip}
\marginparwidth15mm


\title{Notatki z Analizy matematycznej ze źródeł wszelakich}
\author{}
\date{}

\newtheorem{theorem}{Twierdzenie}


\renewcommand*{\proofname}{\textbf{Dowód}}
\numberwithin{equation}{section}

\begin{document}

\begin{titlepage}
\maketitle
\end{titlepage}



\tableofcontents

\section{Liczby naturalne}

\subsection{Przykład indukcji - nierówność Bernoulliego}

\begin{theorem}
    Jeśli $x > -1$ oraz $x \neq 0$ i $n \geq 2$, to:
    \begin{equation*}
        (1 + x)^{n} > 1 + nx.
    \end{equation*}
\end{theorem}

\begin{description}
    \item[Baza indukcji:] Dla $n = 2$ sprawdzamy, czy nierówność Bernoulliego jest prawdziwa. Mamy zatem:
    
    \begin{equation*}
        (1 + x)^{2} > 1 + 2x
    \end{equation*}
    \begin{equation*}
        1 + 2x + x^{2} > 1 + 2x \ \ / \ +(-1 -2x)
    \end{equation*}
    \begin{equation*}
        x^2 > 0
    \end{equation*}
    Z założenia wiemy, że $x \neq 0$, kwadrat liczby rzeczywistej jest zawsze większy niż $0$, więc nierówność jest spełniona.
    \item[Krok indukcyjny] Zakładamy, że $(1 + x)^{n} > 1 + nx$ jest spełnione dla pewnego $n \geq 2$.
   
    \item[Dowodzimy że nierówność jest spełniona dla $n + 1$:] Dowodzimy, że $(1 + x)^{n + 1} > 1 + (n + 1)x$. 
    \begin{equation*}
        (1 + x)^{n + 1} = (1 + x)^{n} (1 + x)^{1}
    \end{equation*}

    Zauważmy, że możemy powyższą informację wykorzystać do odpowiedniego dobrania nierówności, używając nierówności z kroku indukcyjnego (gdy opuścimy
    wyrażenie $(1 + x)$ wracamy po prostu do nierówności z kroku indukcyjnego):
    \begin{equation*}
        (1 + x)^{n + 1} = (1 + x)^{n} (1 + x)^{1} > (1 + nx)(1 + x) = 1 + x + nx + nx^2 = 1 + (n + 1)x + nx^2
    \end{equation*}
    \begin{equation*}
        (1 + x)^{n + 1} > 1 + (n + 1)x + nx^2 > 1 + (n + 1)x
    \end{equation*}
    Jeżeli dodamy "coś" (u nas $nx^2$) do $1 + (n + 1)x$ to i tak widzimy, że $(1 + x)^{n + 1}$ nadal jest większe. Jako, że $x \neq 0$
    (ważne, gdyby było inaczej, nierówność ostra nie byłaby spełniona) to wyrażenie $nx^2$ jest dodatnie. Tym bardziej $(1 + x)^{n + 1}$ jest większe od wyrażenia bez $nx^2$.
    Z tego można wywnioskować końcowy wniosek, który na mocy zasady indukcji pokazuje, że dla $n \geq 2$ nierówność Bernoulliego jest spełniona, czyli
    udowodnione zostało, że dla $n + 1$:

    \begin{equation*}
        (1 + x)^{n + 1} > 1 + (n + 1)x
    \end{equation*}
\end{description}



% TODO: Pobawic się srodowiskiem proof zeby zrobic indukcje
%\begin{proof}
%   
%    \begin{description}
%        \item[Animal] Living being
%      
%        \item[Vegetable] Plant
%       
%        \item[Mineral] Natural inorganic substance
%      \end{description}
%\end{proof}

\section{Własności ciągów liczbowych}
\subsection{Granica iloczynu ciągów jest równa  iloczynowi granic tych ciągów}



\begin{theorem}
    
    Jeżeli $\lim_{n\to\infty} x_n = x$ i $\lim_{n\to\infty} y_n = y$ 
     to $\lim_{n\to\infty} (x_{n}y_{n}) = xy$.

\end{theorem}

\subsection{Dowód, że $\lim_{n\to\infty} (x_{n} + y_{n}) = x + y$}

Niech $x_{n} \;\; \textrm{oraz} \;\; y_{n}$ będą ciągami o wyrazach rzeczywistych  lub zespolonych.

\begin{theorem}
    
    Jeżeli $\lim_{n\to\infty} x_n = x$ i $\lim_{n\to\infty} y_n = y$ 
     to $\lim_{n\to\infty} (x_{n} + y_{n}) = x + y$.

\end{theorem}


\noindent
Chcemy okazać, że dla dowolnej liczby rzeczywistej większej niż zero, istnieje jakaś liczba naturalna, taka, że dla $n$ równego bądź większego niż ta liczba, spełniona jest nierówność:

\begin{equation*}
    |(x_{n} + y_{n}) - (x + y)| < \textrm{Dowolna ustalona liczba rzeczywista większa od zera}
\end{equation*}
\noindent
Niech $\epsilon > 0$ będzie ustaloną liczbą rzeczywistą. Przyjmijmy, że $\delta = \frac{\epsilon}{2}$. Wiemy, że $\delta > 0$. Z założeń wiemy, że oba ciągi są zbieżne, więc dla każdego z tych dwóch ciągów
istnieją liczby naturalne $l, m \in \mathbb{N}$ takie że:


\begin{equation*}
    \bigwedge_{n \geq l} |x_{n} - x| < \delta 
\end{equation*}

\begin{equation*}
    \bigwedge_{n \geq m} |y_{n} - y| < \delta 
\end{equation*}
Oznaczmy $n_0 = \max \{l, m\}$. $n_0$ jest liczbą, która gwarantuje nam, że gdy $n \geq n_0$, od tego miejsca te dwie nierówności
są jednocześnie spełnione. Mamy wtedy:

\begin{equation} \label{suma ciagow polaczona warunkiem i}
    \bigwedge_{n \geq n_0} (|x_{n} - x| < \delta \wedge |y_{n} - y| < \delta)
\end{equation}


Z \ref{suma ciagow polaczona warunkiem i} uzyskujemy:

\begin{equation} \label{mniejsze niz dwa delta}
    \bigwedge_{n \geq n_0} |x_n - x| + |y_n - y| < 2 \delta
\end{equation}

Skorzystamy z nierówności trójkąta i rozwiniemy wyrażenie $|(x_{n} + y_{n}) - (x + y)|$:

\begin{equation*}
    |(x_{n} + y_{n}) - (x + y)| = |(x_{n} - x) + (y_{n} - y)| \leq |x_n - x| + |y_n - y| 
\end{equation*}

Widzimy, że dla dowolnego $n \geq n_0$:

\begin{equation*}
    |(x_{n} + y_{n}) - (x + y)| < 2 \delta = 2 \cdot \frac{\epsilon}{2} = \epsilon
\end{equation*}

\vspace{10mm}
Dla dowolnego $\epsilon > 0$ istnieje takie $n_0$, że $\underset{n \geq n_0}{\bigwedge} |(x_{n} + y_{n}) - (x + y)| < \epsilon$. Zatem $x+y$
jest granicą $(x_{n} + y_{n})$.


\section{Ciągi liczb rzeczywistych}

W tym rozdziale każdy rozważany ciąg jest o wyrazach rzeczywistych.

\begin{theorem}
    
    Ciąg monotoniczny jest zbieżny wtedy i tylko wtedy, gdy jest ograniczony.

\end{theorem}


Każdy ciąg zbieżny jest ograniczony (np. Rudnicki Twierdzenie 3 Punkt 2.1.3). W jedną stronę dowód jest w takim razie gotowy.
Pokażemy zatem, że ciąg który jest monotoniczny i ograniczony jest zbieżny.
\vspace{10mm}


\noindent
Załóżmy, że ciąg $(x_n)$ jest nierosnący (ciąg nazywamy monotonicznym gdy jest niemalejący lub nierosnący) oraz, że jest ograniczony.
Niech: 

\begin{equation*}
    A = \{ x_n : n \in \mathbb{N} \}.
\end{equation*}


\noindent
Zauważmy, że $A$ jest zbiorem ograniczonym. Co to znaczy, że zbiór jest ograniczony? Mówimy, że zbiór $A$ jest ograniczony jeśli
jest ograniczony z góry i z dołu.
\vspace{10mm}

Zbiór $A$ jest \emph{ograniczony z góry} gdy: $$\bigvee_{M \in \mathbb{R}}  \bigwedge_{x \in A} x \leq M$$.

Zbiór $A$ jest \emph{ograniczony z dołu} gdy: $$\bigvee_{m \in \mathbb{R}}  \bigwedge_{x \in A} x \geq m$$.


Wcześniej założyliśmy, że ciąg $(x_n)$ jest ograniczony. Wtedy istnieje liczba dodatnia ograniczająca wszystkie wyrazy tego ciągu. Niech $M > 0$ będzie taką liczbą, 
która dla dowolnego $n$ spełnia nierówność $M \geq |x_n|$. Na mocy pewnej znanej własności dotyczącej wartości bezwzględnej (patrz Wikipedia angielska) mamy:


\begin{equation*}
    M \geq |x_n| \iff -M \leq x_n \leq M
\end{equation*}

\noindent
Ciekawe spostrzeżenie jest takie, że przez założenie o ograniczoności ciągu, widzimy powyżej, że jeśli wrzucimy wyrazy tego ciągu do jakiegoś zbioru
to ten zbiór będzie ograniczony. Jest nam to niezbędne do posłużenia się aksjomatem ciągłości. Wiemy wtedy, że zbiór $A$ ma kres dolny.
Niech zatem $x = \inf A$. Z definicji kresu dolnego mamy wtedy:


\begin{equation*}
    \bigwedge_{n \in \mathbb{N}} x \leq x_n
\end{equation*}

oraz

\begin{equation*}
    \bigwedge_{\epsilon > 0} \bigvee_{x_{n_0} \in A} x_{n_0} < x + \epsilon
\end{equation*}

\noindent
Teraz ważny krok, przypomnijmy, że założyliśmy, że nasz ciąg jest nierosnący. Czyli dla każdego $n \in \mathbb{N}$ $x_{n + 1} \leq x_n$.
Na pierwszy rzut oka może się to nie wydawać oczywiste, ale wykorzystamy tę informację do obsadzenia "własnych" wskaźników.
Prawdziwym jest, że dla $n \geq n_0$ mamy $x_n \leq x_{n_0}$. Dlaczego? Wyrazy ciągu wyglądają jakoś tak:


\begin{equation*}
    x_{1}, x_{2}, x_{3}, x_{4}, x_{5}, x_{6} ...
\end{equation*}


\noindent
Każdy kolejny wyraz ciągu, jest mniejszy niż poprzedni, jeśli wiemy zatem z aksjomatu ciągłości o istnieniu liczby  $x_{n_0}$ to jest gdzieś ona w tym zbiorze wyrazów ciągu:

\begin{equation*}
    x_{1}, x_{2}, x_{3}, x_{4}, x_{5}, x_{6}, x_{7}, x_{8}, x_{9}, x_{10} ... , x_{n_0}, ...
\end{equation*}

\noindent
Czyli od pewnego indeksu wyrazy ciągu są mniejsze bądź równe niż poprzedni wyraz. Jeżeli np. $n_0 = 3$, to wyrazy z indeksem większym bądź równym niż $n = 3$ będa automatycznie mniejsze
na mocy założenia o ciągu niemalejącym:

\begin{equation*}
    x_{1} \geq x_{2} \geq x_{n_0 = 3} \geq x_{4} \geq x_{5} \geq x_{6} \geq x_{7} \geq x_{8} ...
\end{equation*}

\vspace{10mm}
\noindent
W takim razie wykorzystamy tę informację do zapisania nierówności:

\begin{equation*}
    x \leq x_n \leq x_{n_0}  < x + \epsilon \quad \textrm{dla} \quad n \geq n_0
\end{equation*}

\noindent
Z powyższego mamy:

\begin{equation*}
    x_n < x + \epsilon \quad \textrm{dla} \quad n \geq n_0
\end{equation*}

oraz

\begin{equation*}
    x - x_{n} \leq 0 < \epsilon \implies x - \epsilon  < x_{n}  \quad \textrm{dla} \quad n \geq n_0
\end{equation*}


\vspace{10mm}
\noindent
Z tego już łatwo korzystając z własności wartości bezwzględnej mamy:

\begin{equation*}
    x - \epsilon  < x_{n} < x + \epsilon \iff |x_{n} - x| < \epsilon \quad \textrm{dla} \quad n \geq n_0
\end{equation*}


\noindent
Tak więc pokazaliśmy, że dla dowolnego $\epsilon > 0$, istnieje taka liczba (indeks po naszemu) $n_0$, że dla dowolnej liczby $n \geq n_0$
spełniona jest nierówność łudząco przypominająca definicję granicy:

$$  \bigwedge_{n \geq n_0} |x_{n} - x| < \epsilon ,$$

tak więc ciąg monotoniczny ograniczony jest zbieżny do $x$.

\subsection{Granica $\lim_{n\to\infty} \frac{1}{n^\alpha} = 0$ gdy $\alpha > 0$} \label{jedenNadEn}

Sprawdzamy, że: $$\lim_{n\to\infty} \frac{1}{n^ \alpha} = 0 \ \textrm{dla} \ \alpha > 0$$.

\vspace{10mm}

Chcemy sprawdzić, że dla dowolnej liczby $\epsilon > 0$ istnieje takie $n_{0} \in \mathbb{N}$, że:


\begin{equation*}
    |\frac{1}{n^\alpha} - 0| < \epsilon \ \ \textrm{dla} \ \ n \geq n_0
\end{equation*}

\vspace{10mm}

Jako, że wyrażenie $\frac{1}{n^\alpha}$ dla naszych warunków jest zawsze większe niż $0$, możemy opuścić wartość bezwzględną:

\begin{equation*}
    \frac{1}{n^\alpha} < \epsilon \ \ \textrm{dla} \ \ n \geq n_0
\end{equation*}

\vspace{10mm}

Teraz, żeby pokazać, że istnieje liczba od której nierówność z epsilonem jest spełniona, przekształćmy nierówność do równoważnej postaci
tak by $n$ było trochę bardziej "widoczne" tj. dla $n \geq n_0$:

\begin{equation*}
    \frac{1}{n^\alpha} < \epsilon \ \  / \cdot n^\alpha
\end{equation*}

\begin{equation*}
    1 < \epsilon n^\alpha \ \  / \cdot \frac{1}{\epsilon}
\end{equation*}

\begin{equation*}
    \frac{1}{\epsilon} < n^\alpha \ \  / \sqrt[\alpha]{}
\end{equation*}

\begin{equation*}
    n > \big(\frac{1}{\epsilon}\big)^{\frac{1}{\alpha}}
\end{equation*}


Teraz, wiemy, że $\epsilon$ i $\alpha$ jest dowolną liczbą większą niż zero, to oznacza, że jak zostaną ustalone to będziemy mieli
po prawej stronie jakąś liczbę. Zawsze możemy wskazać, liczbę naturalną większą niż ta liczba. Niech w takim razie $n_0$ będzie
liczbą naturalną większą niż $\big(\frac{1}{\epsilon}\big)^{\frac{1}{\alpha}}$ tzn. $n_{0} > \big(\frac{1}{\epsilon}\big)^{\frac{1}{\alpha}}$.
Wskazaliśmy teraz właśnie istnienie liczby, od której (aż wzwyż, czyli liczby powyżej $n_0$ też spełniają tę nierówność) nasza przekształcona nierówność
jest spełniona. Ta nierówność jest oczywiście równoważna $|\frac{1}{n^\alpha} - 0| < \epsilon \ \ \textrm{dla} \ \ n \geq n_0$.
Tak więc granicą dla ciągu $\frac{1}{n^{\alpha}}$ jest liczba $0$.


\subsection{Granica $\lim_{n\to\infty} a^{n} = 0,$ gdy $0 \leq a < 1$}


Postępujemy podobnie jak w granicy ciągu $\frac{1}{n^{\alpha}}$. Chcemy sprawdzić, że $0$ jest granicą ciągu $a^n$ dla warunków $0 \leq a < 1$.
W takim razie sprawdzamy, że dla dowolnej liczby $\epsilon > 0$, istnieje takie $n_0 \in \mathbb{N}$, że

\begin{equation*}
    |a^{n} - 0| < \epsilon \ \ \textrm{dla} \ \ n \geq n_0
\end{equation*}


\noindent
Z podanych warunków, wiemy, że wyrażenie w środku wartości bezwzględnej nie będzie ujemne, możemy ją więc opuścić:

\begin{equation*}
    a^{n}  < \epsilon \ \ \textrm{dla} \ \ n \geq n_0
\end{equation*}

\noindent
Ważne wtrącenie: Dla $a = 0$ ciąg jest stały o wyrazie $0$ już od pierwszego wyrazu, więc ma on granicę równą $0$. W dalszych rozważaniach ograniczmy więc $a$ do przypadku $0 < a < 1$ (pozwoli nam to swobodnie używać
logarytmu, jako, że gdy przyjmiemy za podstawę $a$ mógłby być 'kłopot' z $\epsilon$ w nierówności, z powodu zera w podstawie).
Skorzystajmy z własności logarytmu ($\log _{a} a^b = b$) i dokonajmy przekształcenia powyższej nierówności, aby $n$ było bardziej widoczne, oczywiście
nierówność ta będzie równoważna pierwotnej, zatem dla $n \geq n_0$ zachodzi (Uwaga na zmianę znaku w nierówności!):

\begin{equation*}
    a^{n}  < \epsilon \ \ / \ \ \log _{a}
\end{equation*}

\begin{equation*}
    \log _{a} a^{n} = n > \log _{a} \epsilon 
\end{equation*}

\noindent
Teraz, gdy $\epsilon$ oraz $a$ zostaną ustalone, po prawej stronie powyższej nierówności mamy jakąś skończoną liczbę.
Niech zatem $n_0$ będzie liczbą naturalną taką, że $n_0 > \log _{a} \epsilon$. Właśnie pokazaliśmy, że istnieje liczba (i jako, że jest to liczba naturalna jej następnik etc.) od której
nierówność $n > \log _{a} \epsilon$ jest spełniona, która jest równoważna pierwotnej nierówności. Pokazaliśmy zatem, że dla dowolnego $\epsilon > 0$ istnieje liczba naturalna $n_0$ która dla każdego $n \geq n_0$
spełnia:


\begin{equation*}
    |a^{n} - 0| < \epsilon
\end{equation*}

\vspace{10mm}


więc $\lim_{n\to\infty} a^{n} = 0,$ gdy $0 \leq a < 1$.


\subsection{Granica $\lim_{n\to\infty} \sqrt[n]{a} = 1$ gdy $a > 0$}

Mamy udowodnić, że granica z $\lim_{n\to\infty} \sqrt[n]{a}$ gdy $a > 0$ wynosi $1$. Na starcie zobaczmy co się dzieje w momencie
gdy $a \geq 1$. Weźmy wtedy pierwiastek n-tego stopnia z obydwu stron nierówności, nie musimy uważać na znak nierówności ponieważ $a$ jest dodatnie tak samo jak $1$.
Mamy zatem:

\begin{equation*}
    a \geq 1 / \ \ \sqrt[n]{}
\end{equation*}

\begin{equation*}
    \sqrt[n]{a} \geq \sqrt[n]{1} = 1 
\end{equation*}

\noindent
Weźmy teraz $b_{n} = \sqrt[n]{a} - 1$. Wyliczmy z tej równości $a$:

\begin{equation*}
    b_{n} = \sqrt[n]{a} - 1 \ \ / \ \ +1
\end{equation*}
\begin{equation*}
    b_{n} + 1 = \sqrt[n]{a} \ \ / \ \ (.)^{n}
\end{equation*}
\begin{equation*}
    (b_{n} + 1)^{n} = a 
\end{equation*}

Ważna uwaga, korzystamy z nierówności Bernoulliego, odwołując się do (1.1.1), udowodniliśmy, wersję dla nierówności ostrej, w dowodzie tej granicy wykorzystamy wersję nieostrą:

\begin{equation*}
    a = (b_{n} + 1)^{n} \geq 1 + b_{n}n
\end{equation*}

\noindent
Przekształćmy nierówność, aby wydobyć $b_{n}$:

\begin{equation*}
    a \geq 1 + b_{n}n \ \ / \ \ -1
\end{equation*}
\begin{equation*}
    a - 1 \geq b_{n}n \ \ / \ \ \cdot \frac{1}{n}
\end{equation*}
\begin{equation*}
    \frac{a - 1}{n} \geq b_{n}
\end{equation*}

\noindent
Stworzymy teraz nierówność, która da nam podstawę do użycia twierdzenia o trzech ciągach. Wiemy, że $b_{n} \geq 0$ (dlatego, że $b_{n} = \sqrt[n]{a} - 1 \geq 0$ a $a \geq 1$).
Ustawmy w takim razie następującą nierówność:

\begin{equation*}
    0 \leq b_{n} \leq \frac{a - 1}{n}
\end{equation*}

\noindent
$\lim_{n\to\infty} 0$ wynosi $0$ (na mocy twierdzenia o granicy ciągu stałego). Granica $lim_{n\to\infty} \frac{a - 1}{n}$ jest równa:

\begin{equation*}
    \lim_{n\to\infty} \frac{a - 1}{n} = \lim_{n\to\infty} (a-1) \cdot \frac{1}{n} = \lim_{n\to\infty} (a-1) \cdot  \lim_{n\to\infty} \frac{1}{n} = \big(\lim_{n\to\infty} (a-1)\big) \cdot 0 = 0
\end{equation*}

\noindent
Z twierdzenia o trzech ciągach mamy zatem $\lim_{n\to\infty} b_{n} = 0$. Rezultatu tego użyjemy w wyrażeniu opisującym $b_{n}$:

\begin{equation*}
    \lim_{n\to\infty} b_{n} = \lim_{n\to\infty}(\sqrt[n]{a}) - \lim_{n\to\infty}(1)
\end{equation*}

\begin{equation*}
    0 = \lim_{n\to\infty}(\sqrt[n]{a}) - 1 \ \ / \ \ +1
\end{equation*}

\begin{equation*}
    1 = \lim_{n\to\infty}(\sqrt[n]{a})
\end{equation*}

Dla $a \geq 1$ udowodniliśmy, że granica wynosi $1$. Pozostaje sprawdzić przypadek dla $a \in (0, 1)$. Ale najpierw pokażemy, że wyrażenie $\sqrt[n]{a}$ da się przedstawić trochę inaczej.
Będzie to kluczowe dla dostrzeżenia czemu granica dla tego przypadku również wynosi $1$.

\begin{equation*}
    \sqrt[n]{a} = \sqrt[n]{\frac{1}{\frac{1}{a}}} = \sqrt[n]{(\frac{1}{a})^{-1}} = (\frac{1}{a})^{\frac{-1}{n}} = \frac{1}{(\frac{1}{a})^{\frac{1}{n}}} = \frac{1}{\sqrt[n]{\frac{1}{a}}}
\end{equation*}

\begin{equation*}
    \lim_{n\to\infty}\sqrt[n]{a} = \lim_{n\to\infty}\frac{1}{\sqrt[n]{\frac{1}{a}}} 
\end{equation*}

Ważne spostrzeżenie, $a \in (0, 1)$, powyżej udowodniliśmy, że granica z $\sqrt[n]{a}$ wynosi $1$ gdy $a \geq 1$. Oznacza to dla nas tyle, że w wyrażeniu pod pierwiastkiem jest ułamek, którego wynikiem
jest liczba większa niż $1$. Wiemy zatem jaki jest wynik dla tej granicy, znamy granicę licznika oraz mianownika, separując:

\begin{equation*}
    \lim_{n\to\infty}\sqrt[n]{a} = \frac{\lim_{n\to\infty} 1}{\lim_{n\to\infty} \sqrt[n]{\frac{1}{a}}} = \frac{1}{1} = 1
\end{equation*}

\subsection{Granica $\lim_{n\to\infty} \sqrt[n]{n} = 1$}

Chcemy sprawdzić, że $\lim_{n\to\infty} \sqrt[n]{n} = 1$. Przyjmijmy zatem, że $1 + b_{n} = \sqrt[n]{n}$. Wtedy:

\begin{equation*}
    1 + b_{n} = \sqrt[n]{n} \ \ / \ \ -1
\end{equation*}
\begin{equation*}
    b_{n} = \sqrt[n]{n} - 1 \geq 0
\end{equation*}

\noindent
Skąd wiemy, że $b_{n} \geq 0$? Po pierwsze, wiemy na pewno, że $b_{n}$ może przyjąć $0$ dla $n = 1$ ($b_{n=1} = \sqrt[1]{1} - 1 = 0$). W takim razie co by się stało, gdyby $b_{n} < 0$? Otóż:

\begin{equation*}
    b_{n} = \sqrt[n]{n} - 1 < 0
\end{equation*}
\begin{equation*}
    \sqrt[n]{n} - 1 < 0 \ \ / \ \ +1
\end{equation*}
\begin{equation*}
    \sqrt[n]{n} < 1 \ \ / \ \ (\cdot)^n
\end{equation*}
\begin{equation*}
    (\sqrt[n]{n})^{n} = (n^{\frac{1}{n}})^n = n^1 = n < 1^n = 1
\end{equation*}

\begin{center}
    SPRZECZNOŚĆ! Żadna liczba $n$ nie będzie mniejsza niż $1$.
\end{center}

\noindent
Wiemy już dlaczego na pewno $b_{n} \geq 0$. Wyliczmy, $n$:

\begin{equation*}
    1 + b_{n} = \sqrt[n]{n} \ \ / \ \ (\cdot)^n
\end{equation*}
\begin{equation*}
    (1 + b_{n})^n = n 
\end{equation*}

\noindent
Skorzystamy teraz ze wzoru Newtona na potęgę dwumianu (Wikipedia nazywa ten szczególny przypadek Szeregiem Newton'a? Sprawdzić). Rozwińmy wyrażenie $(1 + b_{n})^n$ tymże wzorem:

\begin{equation*}
    (1 + b_{n})^n = \sum_{k=0}^{n} {n \choose k} b_{n}^{k} = {n \choose 0}b_{n}^{0} + {n \choose 1}b_{n}^{1} + {n \choose 2}b_{n}^{2} + \mbox{...}
\end{equation*}
\begin{equation*}
    \sum_{k=0}^{n} {n \choose k} b_{n}^{k} = 1 + {n \choose 1}b_{n} + {n \choose 2}b_{n}^{2} + \mbox{...}
\end{equation*}

\noindent
Mamy w takim razie dla $n \geq 2$:

\begin{equation*}
    (1 + b_{n})^n \geq 1 + {n \choose 1}b_{n} + {n \choose 2}b_{n}^{2} \geq 1 + {n \choose 2}b_{n}^{2}
\end{equation*}
\begin{equation*}
    (1 + b_{n})^n \geq 1 + {n \choose 2}b_{n}^{2} \ \ / \ \ -1
\end{equation*}
\begin{equation*}
    (1 + b_{n})^n - 1 = n - 1 \geq {n \choose 2}b_{n}^{2}  \ \ / \ \ (:){n \choose 2}
\end{equation*}
\begin{equation*}
    \frac{(n - 1)}{{n \choose 2}} \geq b_{n}^{2} 
\end{equation*}
\begin{equation*}
    \frac{(n - 1)}{{n \choose 2}} \geq b_{n}^{2} 
\end{equation*}
\begin{equation*}
    \frac{(n - 1)}{\frac{n!}{2!(n-2)!}} \geq b_{n}^{2} 
\end{equation*}


Zauważmy, że $\frac{n!}{2!(n-2)!}$ da się zapisać w sposób wygodniejszy do uproszczenia powyższej nierówności:

\begin{equation*}
    \frac{n!}{2!(n-2)!} = \frac{n(n-1)(n-2)!}{(n-2)!2!} =  \frac{n(n-1)}{2}
\end{equation*}

Z tego mamy zatem:

\begin{equation*}
    \frac{(n - 1)}{\frac{n!}{2!(n-2)!}} = \frac{n-1}{\frac{n(n-1)}{2}} = (n-1)\frac{2}{n(n-1)} = \frac{2}{n} \geq b_{n}^{2} \ \ / \ \ \sqrt[2]{}
\end{equation*}
\begin{equation*}
    \frac{\sqrt[]{2}}{\sqrt[]{n}} \geq b_{n}
\end{equation*}

Teraz, pamiętając jak na samym początku pokazaliśmy, że $b_{n} \geq 0$ stwórzmy na tej kanwie bazę pod twierdzenie o trzech ciągach:

\begin{equation*}
    0 \leq b_{n} \leq \frac{\sqrt[]{2}}{\sqrt[]{n}}
\end{equation*}

Z \ref{jedenNadEn} możemy wywnioskować, że (hint: zamień pierwiastek na równoważną wersję z potęgą i wyciąg stałą przed granicę, stałą jest pierwiastek nad 2):
\begin{equation*}
    \lim_{n\to\infty} \frac{\sqrt[]{2}}{\sqrt[]{n}} = 0
\end{equation*}

Podobnie granica ciągu składającego się z samych $0$ jest równa zero. Z twierdzenia o trzech ciągach wnioskujemy, że:

\begin{equation*}
    \lim_{n\to\infty} b_{n} = 0
\end{equation*}

Podsumowując:

\begin{equation*}
    \lim_{n\to\infty} \sqrt[n]{n} = \lim_{n\to\infty} (1 + b_{n}) = 1 + 0 = 1
\end{equation*}

\subsection{Granica $\lim_{n\to\infty} \frac{n^k}{a^n} = 0$, dla $a > 1$, $k \in \mathbb{R}$}



Najpierw sprawdzimy, że $\lim_{n\to\infty} \frac{n}{b^n} = 0$ dla $b > 1$. Niech $b = 1 + c$. Wtedy:

\begin{equation*}
    b = 1 + c > 1 \ \ / \ \ -1
\end{equation*}
\begin{equation*}
    c > 0
\end{equation*}

Rozwińmy, $b^n$ przy użyciu wzoru Newtona:

\begin{equation*}
    b^n = (1 + c)^n = \sum_{k=0}^{n} {n \choose k} c^k = {n \choose 0} c^0 + {n \choose 1} c^1 + {n \choose 2} c^2 \mbox{...}
\end{equation*}

Dla $n \geq 2$:

\begin{equation*}
    b^n \geq {n \choose 0} c^0 + {n \choose 1} c^1 + {n \choose 2} c^2 \geq {n \choose 2} c^2,
\end{equation*}

\begin{equation*}
    b^n \geq {n \choose 2} c^2,
\end{equation*}

\noindent
Teraz przygotujemy grunt pod użycie twierdzenia o trzech ciągach, pamiętając o tym, że jeśli ułamki mają takie same liczniki to większy jest ten który ma mniejszy mianownik.
Poczyńmy również pewną obserwację:

\begin{equation*}
    {n \choose 2} c^2 = \frac{n!}{2!(n - 2)!} \cdot c^2 = \frac{n (n-1) (n - 2)!}{2!(n - 2)!} \cdot c^2 = \frac{n(n-1)c^2}{2}
\end{equation*}

\noindent
Wykorzystamy te informację by lepiej uwidocznić jedno wyrażenie, ustawiając w nierówność:

\begin{equation*}
    0 \leq \frac{n}{b^n} \leq \frac{n}{{n \choose 2} c^2} \leq \frac{n}{{n \choose 2} c^2} = \frac{n}{\frac{n(n-1)c^2}{2}} = \frac{2}{(n-1)c^2}
\end{equation*}

\begin{equation*}
    0 \leq \frac{n}{b^n} \leq \frac{2}{(n-1)c^2}
\end{equation*}

\noindent
Teraz, wiemy, że granica z ciągu $0$ jest równa zero (ciąg stały). A co z granicą $\lim_{n\to\infty} \frac{2}{(n-1)c^2}$?
$c$ jest stałą wyciągnijmy ją zatem przed granicę:

\begin{equation*}
    \lim_{n\to\infty} \frac{2}{(n-1)c^2} = \frac{2}{c^2} \lim_{n\to\infty} \frac{1}{n-1} = \frac{2}{c^2} \cdot 0
\end{equation*}

Korzystając zatem z twierdzenia o trzech ciągach:

\begin{equation*}
    \lim_{n\to\infty} \frac{n}{b^n} = 0 \ \ \ \mbox{Dla} \ \ \ b > 1.
\end{equation*}

\noindent
Wróćmy teraz do głównej granicy którą zamierzaliśmy sprawdzić. Wiemy, że $k \in \mathbb{R}$. Zatem istnieje liczba naturalna większa niż $k$, niech będzie to $p$, $p > k$.
Niech $b = \sqrt[p]{a}$. Zauważmy, że jeśli najpierw podniesiemy $b$ do potęgi $p$ a później do $n$ to $a^n = b^{pn}$ Połóżmy pewną nierówność jako grunt pod użycie twierdzenia o trzech ciągach:

\begin{equation*}
    0 \leq \frac{n^k}{a^n} \leq \frac{n^p}{a^n} = \frac{n^p}{b^{pn}} = \Big(\frac{n}{b^n}\Big)^p
\end{equation*}

\noindent
Wiemy, że dla potęgi naturalnej $p$ i $\lim_{n\to\infty} x_{n} = x$ mamy $\lim_{n\to\infty} x_{n}^p = x^p$. Z tego zatem mamy, że:

\begin{equation*}
    \lim_{n\to\infty} \Big(\frac{n}{b^n}\Big)^p = 0^p = 0
\end{equation*}

\noindent
Z twierdzenia o trzech ciągach dostajemy granicę równą $0$ dla ciągu $\frac{n^k}{a^n}$ o ile $a > 1$ a $k \in \mathbb{R}$.

\subsection{Granica z $\lim_{n\to\infty} x_n^p = 0^p = 0$ gdy $x_n \xrightarrow[n \to \infty]{} 0$}


Niech $x_n$ będzie ciągiem liczb rzeczywistych takim, że  $x_n \geq 0$ dla $n \in \mathbb{N}$. Niech $p > 0$.  
Jeśli $\lim_{n\to\infty} x_n = 0$ to wtedy $\lim_{n\to\infty} x_n^p = 0$.

\vspace{10mm}
\noindent

Ustalmy $\epsilon > 0$. Wówczas możemy przyjąć, że $\delta = \epsilon^{1/p}$. Zatem $\delta > 0$ i istnieje liczba $n_0$ która dla $n \geq n_0$
spełnia następującą nierówność:

\begin{equation*}
    |x_n - 0| = x_n < \delta \ \ \ / \ \ (.)^{(p)}
\end{equation*}
Przy potęgowaniu znaku nierówności nie musimy zmieniać, ponieważ po lewej i po prawej nie będzie wartości ujemnych:
\begin{equation*}
    x_n^p < \delta^p = \Big(\epsilon^{1/p}\Big)^p = \epsilon
\end{equation*}
Zatem dla dowolnego $\epsilon$ istnieje $n_0$, które dla $n \geq n_0$ spełnia $|x_{n}^{p} - 0| < \epsilon$.


\subsection{Przykłady jak liczyć granice ciągów - Rudnicki}

\subsubsection{Granica z ilorazu wielomianu i.e $\frac{P(n)}{n^k}$}


Niech $P$ będzie wielomianem stopnia $k$, $P(n) = a_{0} + a_{1}n + \mbox{...} + a_{k}n^{k}$, wtedy:

\begin{equation*}
    \lim_{n\to\infty} \frac{P(n)}{n^k} = \lim_{n\to\infty} \frac{a_{0} + a_{1}n + \mbox{...} + a_{k}n^{k}}{n^k} = \lim_{n\to\infty} \frac{a_{0}}{n^k} + \frac{a_{1}n}{n^k} + \mbox{...} + \frac{a_{k}n^{k}}{n^k} =
\end{equation*}

\begin{equation*}
    = \lim_{n\to\infty} a_{0}n^{-k} + a_{1}n^{1-k} + a_{2}n^{2-k} + \mbox{...} + a_{k} 
\end{equation*}

Warto zauważyć, że możemy użyć twierdzenia \ref{jedenNadEn} aby obliczyć tę granicę. $k$ jest stopniem wielomianu i jest to nieujemna liczba całkowita, zatem każde $n$ ma wykładnik ujemny poza ostatnim, ($k > k - 1 > k - 2$ etc.).
Wobec tego biorąc odwrotności z tych wyrażeń (pamiętamy, potęga ujemna) mamy po prostu sumowanie zer AŻ do $a_k$ kiedy to $n^k$ się skraca do jedności:

\begin{equation*}
    \lim_{n\to\infty} a_{0}n^{-k} + a_{1}n^{1-k} + a_{2}n^{2-k} + \mbox{...} + a_{k} = 0 + 0 + 0 + \mbox{...} + a_k = a_k.
\end{equation*}

\subsubsection{Granica z ilorazu wielomianów i.e $\frac{P(n)}{Q(n)}$}

Niech $P$ będzie wielomianem stopnia $k$, $P(n) = a_{0} + a_{1}n + \mbox{...} + a_{k}n^{k}$ oraz $Q(n) = b_{0} + b_{1}n + \mbox{...} + b_{l}n^{l}$, chcemy oczywiście aby współczynniki przy najwyższch stopniach nie sprawiały, że 
będziemy dzielić przez zero przy liczeniu granicy, więc również $a_k \neq 0$ oraz $b_l \neq 0$ i (kluczowe) $k \leq l$. Użyjemy poprzedniego przykładu i będziemy sprytni. Rozbijemy ułamek wielomianów, tak
aby dzielić $P(n)$ przez $n^k$ i dzielić $n^l$ przez $Q(n)$. Całość konstrukcji musi być oczywiście równa $\frac{P(n)}{Q(n)}$, dlatego  pomnożymy wszystko przez $n^{k-l}$:

\begin{equation*}
    \lim_{n\to\infty} \frac{P(n)}{Q(n)} = \lim_{n\to\infty} \frac{P(n)}{n^k} \frac{n^l}{Q(n)} \cdot n^{k-l}
\end{equation*}

Korzystamy ze wzoru na arytmetykę granic przy iloczynie. $\frac{P(n)}{n^k}$ wiemy z przykładu wcześniej, że granica z tego wyrażenia to $a_k$.
\vspace{5mm}
Teraz poczynimy uwagę, wiemy, że: 

\begin{equation*}
    \lim_{n\to\infty} \frac{Q(n)}{n^l} = b_l
\end{equation*}

Jest nam również znany wzór na granicę ciągu do ujemnej potęgi naturalnej, tj:

\begin{equation*}
    \lim_{n\to\infty} (\frac{Q(n)}{n^l})^{-1} = b_l^{-1} = \frac{1}{b_l}
\end{equation*}

Wzór oczywiście działa, bo "oryginalna" granica wynosi $b_l$, a ciąg dla dowolnego $\mathbb{N}$ nie jest równy zero. Mamy wtedy prostą drogę do podsumowania granicy ilorazu wielomianów.

\begin{equation*}
    \lim_{n\to\infty} \frac{P(n)}{Q(n)} = \lim_{n\to\infty} \frac{P(n)}{n^k} \frac{n^l}{Q(n)} \cdot n^{k-l} = \frac{a_k}{b_l} \cdot \lim_{n\to\infty} n^{k-l}.
\end{equation*}

\vspace{5mm}

Teraz, zastanówmy się jaka może być granica z $\lim_{n\to\infty} n^{k-l}$. Z naszych założeń, $k = l$ albo $k < l$. Gdy $k=l$, sprawa jest prosta, mamy po prostu:

\begin{equation*}
    \lim_{n\to\infty} n^{l-l} = \lim_{n\to\infty} n^{0} = \lim_{n\to\infty} 1 = 1
\end{equation*}

I wtedy: 

\begin{equation*}
    \frac{a_k}{b_l} \cdot \lim_{n\to\infty} n^{k-l} = \frac{a_k}{b_l} \cdot 1 = \frac{a_k}{b_l}.
\end{equation*}


Gdy $k < l$ mamy, ze wzoru \ref{jedenNadEn}:

\begin{equation*}
    \frac{a_k}{b_l} \cdot \lim_{n\to\infty} n^{k-l} = \frac{a_k}{b_l} \cdot 0 = 0
\end{equation*}

Podsumowując:

\begin{equation*}
    \lim_{n\to\infty} \frac{P(n)}{Q(n)} =
    \begin{cases}
        \frac{a_k}{b_l}, \ \ \mbox{gdy} \ \ k = l, \\
        0, \ \ \mbox{gdy} \ \ k < l.
    \end{cases}
\end{equation*}

\end{document}