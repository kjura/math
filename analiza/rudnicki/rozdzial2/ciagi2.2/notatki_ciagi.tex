%\documentclass[12pt, letterpaper]{article}
\documentclass[a4paper,oneside,openright,11pt]{article}
\usepackage[utf8]{inputenc}
\usepackage[T1]{fontenc}
\usepackage{amsmath}
\usepackage{amsfonts}
%\usepackage[a4paper, total={6in, 8in}]{geometry}

\textwidth\paperwidth
\advance\textwidth -55mm
\oddsidemargin-1in
\advance\oddsidemargin 30mm
\evensidemargin-1in
\advance\evensidemargin 25mm
\topmargin -1in
\advance\topmargin 2cm
\setlength\textheight{48\baselineskip}
\addtolength\textheight{\topskip}
\marginparwidth15mm




\newtheorem{theorem}{Twierdzenie}

\title{Rozdział 2 Ciągi 2.2}
\author{}
\date{}

\numberwithin{equation}{section}

\begin{document}

\begin{titlepage}
\maketitle
\end{titlepage}


\section{Własności ciągów liczbowych}
\subsection{Granica iloczynu ciągów jest równa  iloczynowi granic tych ciągów}



\begin{theorem}
    
    Jeżeli $\lim_{n\to\infty} x_n = x$ i $\lim_{n\to\infty} y_n = y$ 
     to $\lim_{n\to\infty} (x_{n}y_{n}) = xy$.

\end{theorem}

\subsection{Dowód, że $\lim_{n\to\infty} (x_{n} + y_{n}) = x + y$}

Niech $x_{n} \;\; \textrm{oraz} \;\; y_{n}$ będą ciągami o wyrazach rzeczywistych  lub zespolonych.

\begin{theorem}
    
    Jeżeli $\lim_{n\to\infty} x_n = x$ i $\lim_{n\to\infty} y_n = y$ 
     to $\lim_{n\to\infty} (x_{n} + y_{n}) = x + y$.

\end{theorem}


\noindent
Chcemy okazać, że dla dowolnej liczby rzeczywistej większej niż zero, istnieje jakaś liczba naturalna, taka, że dla $n$ równego bądź większego niż ta liczba, spełniona jest nierówność:

\begin{equation*}
    |(x_{n} + y_{n}) - (x + y)| < \textrm{Dowolna liczba rzeczywista większa od zera}
\end{equation*}
\noindent
Niech $\epsilon > 0$ będzie ustaloną liczbą rzeczywistą. Przyjmijmy, że $\delta = \frac{\epsilon}{2}$. Wiemy, że $\delta > 0$. Z założeń wiemy, że oba ciągi są zbieżne, więc dla każdego z tych dwóch ciągów
istnieją liczby naturalne $l, m \in \mathbb{N}$ takie że:


\begin{equation*}
    \bigwedge_{n \geq l} |x_{n} - x| < \delta 
\end{equation*}

\begin{equation*}
    \bigwedge_{n \geq m} |y_{n} - y| < \delta 
\end{equation*}
Oznaczmy $n_0 = \max \{l, m\}$. $n_0$ jest liczbą, która gwarantuje nam, że gdy $n \geq n_0$, od tego miejsca te dwie nierówności
są jednocześnie spełnione. Mamy wtedy:

\begin{equation} \label{suma ciagow polaczona warunkiem i}
    \bigwedge_{n \geq n_0} (|x_{n} - x| < \delta \wedge |y_{n} - y| < \delta)
\end{equation}


Z \ref{suma ciagow polaczona warunkiem i} uzyskujemy:

\begin{equation} \label{mniejsze niz dwa delta}
    \bigwedge_{n \geq n_0} |x_n - x| + |y_n - y| < 2 \delta
\end{equation}

Skorzystamy z nierówności trójkąta i rozwiniemy wyrażenie $|(x_{n} + y_{n}) - (x + y)|$:

\begin{equation*}
    |(x_{n} + y_{n}) - (x + y)| = |(x_{n} - x) + (y_{n} - y)| \leq |x_n - x| + |y_n - y| 
\end{equation*}

Widzimy, że dla dowolnego $n \geq n_0$:

\begin{equation*}
    |(x_{n} + y_{n}) - (x + y)| < 2 \delta = 2 \cdot \frac{\epsilon}{2} = \epsilon
\end{equation*}

\vspace{10mm}
Dla dowolnego $\epsilon > 0$ istnieje takie $n_0$, że $\underset{n \geq n_0}{\bigwedge} |(x_{n} + y_{n}) - (x + y)| < \epsilon$. Zatem $x+y$
jest granicą $(x_{n} + y_{n})$.

\end{document}