\documentclass[a4paper,oneside,openright,11pt]{article}
\usepackage[utf8]{inputenc}
\usepackage[T1]{fontenc}
\usepackage{amsmath}
\usepackage{amsfonts}
%\usepackage[a4paper, total={6in, 8in}]{geometry}

\textwidth\paperwidth
\advance\textwidth -55mm
\oddsidemargin-1in
\advance\oddsidemargin 30mm
\evensidemargin-1in
\advance\evensidemargin 25mm
\topmargin -1in
\advance\topmargin 2cm
\setlength\textheight{48\baselineskip}
\addtolength\textheight{\topskip}
\marginparwidth15mm




\newtheorem{theorem}{Twierdzenie}

\title{Rozdział 2 Ciągi 2.2}
\author{}
\date{}

\numberwithin{equation}{section}

\begin{document}

\begin{titlepage}
\maketitle
\end{titlepage}


\section{Własności ciągów liczbowych}
\subsection{Granica iloczynu ciągów jest równa  iloczynowi granic tych ciągów}



\begin{theorem}
    
    Jeżeli $\lim_{n\to\infty} x_n = x$ i $\lim_{n\to\infty} y_n = y$ 
     to $\lim_{n\to\infty} (x_{n}y_{n}) = xy$.

\end{theorem}

\subsection{Dowód, że $\lim_{n\to\infty} (x_{n} + y_{n}) = x + y$}

Niech $x_{n} \;\; \textrm{oraz} \;\; y_{n}$ będą ciągami o wyrazach rzeczywistych  lub zespolonych.

\begin{theorem}
    
    Jeżeli $\lim_{n\to\infty} x_n = x$ i $\lim_{n\to\infty} y_n = y$ 
     to $\lim_{n\to\infty} (x_{n} + y_{n}) = x + y$.

\end{theorem}


\noindent
Chcemy okazać, że dla dowolnej liczby rzeczywistej większej niż zero, istnieje jakaś liczba naturalna, taka, że dla $n$ równego bądź większego niż ta liczba, spełniona jest nierówność:

\begin{equation*}
    |(x_{n} + y_{n}) - (x + y)| < \textrm{Dowolna liczba rzeczywista większa od zera}
\end{equation*}
\noindent
Niech $\epsilon > 0$ będzie ustaloną liczbą rzeczywistą. Przyjmijmy, że $\delta = \frac{\epsilon}{2}$. Wiemy, że $\delta > 0$. Z założeń wiemy, że oba ciągi są zbieżne, więc dla każdego z tych dwóch ciągów
istnieją liczby naturalne $l, m \in \mathbb{N}$ takie że:


\begin{equation*}
    \bigwedge_{n \geq l} |x_{n} - x| < \delta 
\end{equation*}

\begin{equation*}
    \bigwedge_{n \geq m} |y_{n} - y| < \delta 
\end{equation*}
Oznaczmy $n_0 = \max \{l, m\}$. $n_0$ jest liczbą, która gwarantuje nam, że gdy $n \geq n_0$, od tego miejsca te dwie nierówności
są jednocześnie spełnione. Mamy wtedy:

\begin{equation} \label{suma ciagow polaczona warunkiem i}
    \bigwedge_{n \geq n_0} (|x_{n} - x| < \delta \wedge |y_{n} - y| < \delta)
\end{equation}


Z \ref{suma ciagow polaczona warunkiem i} uzyskujemy:

\begin{equation} \label{mniejsze niz dwa delta}
    \bigwedge_{n \geq n_0} |x_n - x| + |y_n - y| < 2 \delta
\end{equation}

Skorzystamy z nierówności trójkąta i rozwiniemy wyrażenie $|(x_{n} + y_{n}) - (x + y)|$:

\begin{equation*}
    |(x_{n} + y_{n}) - (x + y)| = |(x_{n} - x) + (y_{n} - y)| \leq |x_n - x| + |y_n - y| 
\end{equation*}

Widzimy, że dla dowolnego $n \geq n_0$:

\begin{equation*}
    |(x_{n} + y_{n}) - (x + y)| < 2 \delta = 2 \cdot \frac{\epsilon}{2} = \epsilon
\end{equation*}

\vspace{10mm}
Dla dowolnego $\epsilon > 0$ istnieje takie $n_0$, że $\underset{n \geq n_0}{\bigwedge} |(x_{n} + y_{n}) - (x + y)| < \epsilon$. Zatem $x+y$
jest granicą $(x_{n} + y_{n})$.


\section{Ciągi liczb rzeczywistych}

W tym rozdziale każdy rozważany ciąg jest o wyrazach rzeczywistych.

\begin{theorem}
    
    Ciąg monotoniczny jest zbieżny wtedy i tylko wtedy, gdy jest ograniczony.

\end{theorem}


Każdy ciąg zbieżny jest ograniczony (np. Rudnicki Twierdzenie 3 Punkt 2.1.3). W jedną stronę dowód jest w takim razie gotowy.
Pokażemy zatem, że ciąg który jest monotoniczny i ograniczony jest zbieżny.
\vspace{10mm}


\noindent
Załóżmy, że ciąg $(x_n)$ jest nierosnący (ciąg nazywamy monotonicznym gdy jest niemalejący lub nierosnący) oraz, że jest ograniczony.
Niech: 

\begin{equation*}
    A = \{ x_n : n \in \mathbb{N} \}.
\end{equation*}


\noindent
Zauważmy, że $A$ jest zbiorem ograniczonym. Co to znaczy, że zbiór jest ograniczony? Mówimy, że zbiór $A$ jest ograniczony jeśli
jest ograniczony z góry i z dołu.
\vspace{10mm}

Zbiór $A$ jest \emph{ograniczony z góry} gdy: $$\bigvee_{M \in \mathbb{R}}  \bigwedge_{x \in A} x \leq M$$.

Zbiór $A$ jest \emph{ograniczony z dołu} gdy: $$\bigvee_{m \in \mathbb{R}}  \bigwedge_{x \in A} x \geq m$$.


Wcześniej założyliśmy, że ciąg $(x_n)$ jest ograniczony. Wtedy istnieje liczba dodatnia ograniczająca wszystkie wyrazy tego ciągu. Niech $M > 0$ będzie taką liczbą, 
która dla dowolnego $n$ spełnia nierówność $M \geq |x_n|$. Na mocy pewnej znanej własności dotyczącej wartości bezwzględnej (patrz Wikipedia angielska) mamy:


\begin{equation*}
    M \geq |x_n| \iff -M \leq x_n \leq M
\end{equation*}

\noindent
Ciekawe spostrzeżenie jest takie, że przez założenie o ograniczoności ciągu, widzimy powyżej, że jeśli wrzucimy wyrazy tego ciągu do jakiegoś zbioru
to ten zbiór będzie ograniczony. Jest nam to niezbędne do posłużenia się aksjomatem ciągłości. Wiemy wtedy, że zbiór $A$ ma kres dolny.
Niech zatem $x = \inf A$. Z definicji kresu dolnego mamy wtedy:


\begin{equation*}
    \bigwedge_{n \in \mathbb{N}} x \leq x_n
\end{equation*}

oraz

\begin{equation*}
    \bigwedge_{\epsilon > 0} \bigvee_{x_{n_0} \in A} x_{n_0} < x + \epsilon
\end{equation*}

\noindent
Teraz ważny krok, przypomnijmy, że założyliśmy, że nasz ciąg jest nierosnący. Czyli dla każdego $n \in \mathbb{N}$ $x_{n + 1} \leq x_n$.
Na pierwszy rzut oka może się to nie wydawać oczywiste, ale wykorzystamy tę informację do obsadzenia "własnych" wskaźników.
Prawdziwym jest, że dla $n \geq n_0$ mamy $x_n \leq x_{n_0}$. Dlaczego? Wyrazy ciągu wyglądają jakoś tak:


\begin{equation*}
    x_{1}, x_{2}, x_{3}, x_{4}, x_{5}, x_{6} ...
\end{equation*}


\noindent
Każdy kolejny wyraz ciągu, jest mniejszy niż poprzedni, jeśli wiemy zatem z aksjomatu ciągłości o istnieniu liczby  $x_{n_0}$ to jest gdzieś ona w tym zbiorze wyrazów ciągu:

\begin{equation*}
    x_{1}, x_{2}, x_{3}, x_{4}, x_{5}, x_{6}, x_{7}, x_{8}, x_{9}, x_{10} ... , x_{n_0}, ...
\end{equation*}

\noindent
Czyli od pewnego indeksu wyrazy ciągu są mniejsze bądź równe niż poprzedni wyraz. Jeżeli np. $n_0 = 3$, to wyrazy z indeksem większym bądź równym niż $n = 3$ będa automatycznie mniejsze
na mocy założenia o ciągu niemalejącym:

\begin{equation*}
    x_{1} \geq x_{2} \geq x_{n_0 = 3} \geq x_{4} \geq x_{5} \geq x_{6} \geq x_{7} \geq x_{8} ...
\end{equation*}

\vspace{10mm}
\noindent
W takim razie wykorzystamy tę informację do zapisania nierówności:

\begin{equation*}
    x \leq x_n \leq x_{n_0}  < x + \epsilon \quad \textrm{dla} \quad n \geq n_0
\end{equation*}

\noindent
Z powyższego mamy:

\begin{equation*}
    x_n < x + \epsilon \quad \textrm{dla} \quad n \geq n_0
\end{equation*}

oraz

\begin{equation*}
    x - x_{n} \leq 0 < \epsilon \implies x - \epsilon  < x_{n}  \quad \textrm{dla} \quad n \geq n_0
\end{equation*}


\vspace{10mm}
\noindent
Z tego już łatwo korzystając z własności wartości bezwzględnej mamy:

\begin{equation*}
    x - \epsilon  < x_{n} < x + \epsilon \iff |x_{n} - x| < \epsilon \quad \textrm{dla} \quad n \geq n_0
\end{equation*}


\noindent
Tak więc pokazaliśmy, że dla dowolnego $\epsilon > 0$, istnieje taka liczba (indeks po naszemu) $n_0$, że dla dowolnej liczby $n \geq n_0$
spełniona jest nierówność łudząco przypominająca definicję granicy:

$$  \bigwedge_{n \geq n_0} |x_{n} - x| < \epsilon ,$$

tak więc ciąg monotoniczny ograniczony jest zbieżny do $x$.

\end{document}