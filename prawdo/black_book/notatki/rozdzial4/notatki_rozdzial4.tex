\documentclass[12pt, letterpaper, twoside]{article}
\usepackage[utf8]{inputenc}
\usepackage[T1]{fontenc}
\usepackage{amssymb}
\usepackage{amsmath}
\newtheorem{theorem}{Twierdzenie}

\title{Notatki do rozdziału 4 Rachunek prawdopodobieństwa dla (prawie) każdego}
\author{}
\date{}

\begin{document}

\begin{titlepage}
\maketitle
\end{titlepage}


\section{4.1 Definicja. Rozkład zmiennej losowej.}
\subsection{Dowód że przedział $[a,b] = \cap_{n=1}^{\infty}(a - \frac{1}{n}, b]$}

Niech $x \in [a,b]$, $a,b$ $\in \mathbb{R}$. Zgodnie z definicją przedziału
$a \leq x \leq b$. Zauważmy, że $a > a - \frac{1}{n}$ dla $n \in \mathbb{N}$. W takim razie
$x > a - \frac{1}{n}$ dla $n \in \mathbb{N}$. Zapisujemy $x$ pomiędzy
dwiema wielkościami:

\begin{equation*}
    a - \frac{1}{n} < x \leq b  \quad \textrm{dla} \quad n \in \mathbb{N}
\end{equation*}

\noindent
Jako przedział:

\begin{equation*}
    x \in (a - \frac{1}{n}, b] \quad \textrm{dla} \quad n \in \mathbb{N} \iff \cap_{n=1}^{\infty}(a - \frac{1}{n}, b]
\end{equation*}

\vspace{10mm}
\noindent
W drugą stronę. Niech $x \in \cap_{n=1}^{\infty}(a - \frac{1}{n}, b]$

\end{document}